\chapter{ワークC データ取得・分析}

\section{はじめに} 
本ワークにおいて,何を目的にどのような実験を行ったのか,得られた結果から何が考察できるのかについて報告する内容を整理して記述する.
例えば次のように記述する.\\

(例)本ワークでは~を目的とし,~の実験を行った.
実験を通じて~の結果を得た.
それらの結果より,~であることを明らかにする.

\section{方法} 
本ワークで行った実験に関する条件や学んだ理論について記述する.
以下の内容が記載されているか注意しながら進めること.
例文は不慣れな学生向けの一例にすぎず,必ずしも同様の書き方をする必要はない.
必要に応じて図表や数式を用いて説明することが求められる.

\begin{itemize} 
    \item[・] 実験に用いた材料およびその性能,型式など\ (例)○○の評価では,○○(型式: ○○),○○(型式: ○○),および○○(型式: ○○)を図〇〇に示すように使用した. 
    \item[・] データの取得方法および作業手順\ (例)○○データの取得のため,○○を用いて○○を測定した.まず〇〇秒間〇〇し,次に〇〇し$\cdots$,最後に〇〇した. 
    \item[・] 評価方法\ (例)取得されたデータを〇〇法により〇〇し,〇〇を評価した. \end{itemize}

以下のように記述する.

(例)本ワークでは$\cdots$を目的に$\cdots$を行った.
○○の評価では,光センサとして○○と,○○,および○○を図〇〇に示すように配線した.
また,○○の評価のため,○○を行った.○○のデータ取得のため,まず〇〇秒間〇〇し,次に〇〇し$\cdots$最後に〇〇した.
取得されたデータを〇〇法により〇〇し,〇〇を評価した.

\subsection{ワークC-1 LEDマトリクスの活用}

\subsection{ワークC-2 ネットワークを介したデータの取得と分析}

\subsection{ワークC-3 時系列データの平滑化}

\clearpage \section{結果} 
実験から得られた結果について記述し,その中で読者の興味を引く結果がどれであり,それが何であるかを述べる.
グラフや表にまとめるべきデータがあれば適切に図表として示すこと.
本節の目的は図表を活用して要点を簡潔にまとめることであり,図表を文章にするのではない点に注意する.

\subsection{ワークC-1 LEDマトリクスの活用}

\subsection{ワークC-2 ネットワークを介したデータの取得と分析}

\subsection{ワークC-3 時系列データの平滑化}

\clearpage \section{考察} 
本ワークで得られた結果について考察する.
例えば,以下のように記述する.

(例)本ワーク○○では○○の結果が得られたが,これは理論値と比較して○○な結果である.これは○○が原因であると考えられる.


\section{おわりに} 
ワークを通じて得られた知見や考察,目的の達成度などについて記述し,本ワークを総括する.
単なる感想ではなく,客観的な総括を行うように注意すること.

\clearpage

